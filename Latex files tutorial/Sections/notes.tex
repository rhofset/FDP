\newpage

\section{Notes} \label{notes}
It is recommended that you make notes during the tutorial. This will improve what you remember after the tutorial, and it can be used as an encyclopedia later in your studies or career. When you are making notes, you are forcing your brain to think about what you are writing, and therefore you will improve your learning outcomes. How to make notes is up to you, but some tips can be:
\begin{itemize}
    \item Make notes about what \Rtext{\textbf{you}}\ think is important.
    \item Make notes about how you configured features/services.
    \item Make notes about settings you are using that could be important later. For example, IP addresses, passwords.
    \item Make notes about thoughts that you get during the tutorial.
    \item Always use the same format on your notes.
\end{itemize}

If you have trouble making notes during the tutorial, create a note with keywords and rewrite it to something you understand when you have finished with the task.

The notes can be done using pen and paper or using electronic aids such as a computer or an electronic pad. Some examples of computer tools that can help you with this are, for example, Endnote\footnote{\url{https://endnote.com/}}, Obsidian\footnote{\url{https://obsidian.md/}}, Joplin\footnote{\url{https://joplinapp.org/}}, or OneNote\footnote{\url{https://www.microsoft.com/nb-no/microsoft-365/onenote/digital-note-taking-app}}.

\warnblock{If you decide to use some of the tools listed for your notes, you are on your own. This tutorial will not provide guidance for them. If you get problems with them, check the vendor's page or use google to see if other persons have had the same problem.}

\quesblock{\begin{enumerate}
    \item[1.] How do you think making notes during the tutorial will benefit you?
\end{enumerate}}