\newpage{}
\pagestyle{fancy}
\section*{Legend} % non numbered
\addcontentsline{toc}{section}{Legend}
%We are trying a new tutorial template this year. To highlight the key areas in the document where you may need to download, read or answer questions we will make use of the following symbols/icons
The legends will inform you about key details in the tutorial. Please follow them. Be careful when the warning legends are present, orange or red colored.
\setupblock{You will need to do the following configuration}
\vspace{-0.8cm}
\tipbox{You may find this useful to answer the question, to get passed the current task, or just generally}
\vspace{-0.8cm}
%\helpblock{This is extra helpful information if you get stuck}
\dlblock{You will need to download something from somewhere}{http://somewhere.com}
\vspace{-0.8cm}
\quesblock{You should answer this question}
\vspace{-0.8cm}
\nukeblock{This will take a lot of time, but you will find it useful in addition to your study}
\vspace{-0.8cm}
%\blogblock{You should specifically reflect/answer this in your journal}
%broken \importantblock{This is something you need to pay attention to}
\readblock{This is additional reading that will support you}
\vspace{-0.8cm}
\Winblock{These instructions apply to Microsoft Windows}
\vspace{-0.8cm}
\Macblock{These instructions apply to Apple MacOS}
\vspace{-0.8cm}
\Linblock{These instructions apply to a Linux System}
\vspace{-0.8cm}
\gitblock{Grab this project or resource from the following}{https://github.com/rhofset}
\vspace{-0.8cm}
\warnblock{\textbf{This is a \Rtext{POTENTIALLY RISKY}{ }task. }}
\vspace{-0.8cm}
\checkblock{You can do this to know you are correct}
%\bombblock{Stuff goes wrong here}
\vspace{-0.8cm}
\begin{importantblock}
Do NOT do this.
\end{importantblock}
\vspace{0.8cm}

Those legends are here to guide you through the tutorial.
\pline

In the tutorial, you will see something like this when you are configuring the different features: \cmd{System -> Access -> Users}. When you see something like this, it means that you go to the \cmd{System} menu, then to the submenu \cmd{Access}, and at last another sub submenu \cmd{Users} or a tab called \cmd{Users}.
%We hope you find these useful. This block may expanded or removed as we go on during the year.
\pline